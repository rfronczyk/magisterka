% this file is called up by thesis.tex
% content in this file will be fed into the main document

%: ----------------------- name of chapter  -------------------------
\chapter{Analiza wymagań} % top level followed by section, subsection


%: ----------------------- paths to graphics ------------------------

% change according to folder and file names
\ifpdf
    \graphicspath{{3/figures/PNG/}{3/figures/PDF/}{3/figures/}}
\else
    \graphicspath{{3/figures/EPS/}{3/figures/}}
\fi

%: ----------------------- contents from here ------------------------

Tem rodział opisuje wymagania które powinny być spełniane przez poszczególne, przykładowe aplikacje korzystające z serwisu UniversalSynthesizer. Podstawową funkcją serwisu jest udostępnianie usługi zamiany tekstu na mowę. Zaletą niniejszego systemu jest jego otwartość, możliwość wykorzystania innych niż proponowane syntezatorów mowy, łatwy dostęp przez web service, stosunkowo prosty sposób rozbudowy oraz możliwość jego wykorzystania do różnych celów na różnych platformach. Cechy te należy zademonstrować na przykładzie kilku prostych aplikacji, które muszą spełniać pewne jasno określone wymagania funkcjonalne i niefunkcjonalne opisane poniżej. 
\section{Wymagania funkcjonalne}

\subsection{Automatyczne dyktando}
Jest to prosta aplikacja dostępna dla użytkowników w formacie strony internetowej. Jej głównym celem jest wykorzystanie funkcji UniversalSynthesizer do dyktowania treści którą użytkownik zapisuje. Po odczytaniu całego tekstu i wysłaniu przez użytkownika formularza, jest on porównywany z tekstem oryginalnym w ten sposób może ćwiczyć swoją ortografie. Duża zaletą jest to, że aplikacja ta nie jest ograniczona do jednego języka oraz fakt, że użytkownik sam podaje tekst.
\subsubsection {Przypadki użycia} 
\includegraphics[scale=0.55]{useCaseDictando.png} 
\subsubsection{Wymagania funkcjonalne}
\begin{enumerate}
	\item Obsługiwanie więcej niż jednego języka
		\begin{enumerate}
			\item polski
			\item angielski
			\item francuski
		\end{enumerate}
	\item Możliwość podania tekstu źródłowego
		\begin{enumerate}
			\item w formie tekstu wpisywanego na stronie
			\item jako plik tekstowy, wysyłany na serwer
		\end{enumerate}
	\item Umożliwienie użytkownikowi odsłuchania wysłanego tekstu w wybranym przez niego języku
	\item Umożliwienie użytkownikowi wprowadzania tekstu jednocześnie z odsłuchiwaniem
	\item Wygenerowanie i wyświetlenie raportu o ilości błędów popełnionych przez użytkownika
\end{enumerate}




% ---------------------------------------------------------------------------
%: ----------------------- end of thesis sub-document ------------------------
% ---------------------------------------------------------------------------

