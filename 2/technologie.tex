% this file is called up by thesis.tex
% content in this file will be fed into the main document

\chapter{Technologie} % top level followed by section, subsection


% ----------------------- contents from here ------------------------

\section{Apache ServiceMix}
Integracja systemów informatycznych jest jednym z największych wyzwań stojących przed nowoczesnymi przedsiębiorstwami. Apache ServiceMix pomaga rozwiązać ten problem będąc bazującym na standardach, lekkim oraz stosującym paradygmat "luźnego powiązania" narzędziem. Dzięki bazowaniu na standardch w sposób drastyczny zmniejsza szanse na uzależnienie się od konkretnego dostawcy oprogramowania, przez sotowanie luźnego-powiązania zmniejsza złożoność integracji. 
Jest to otwarta implementacja ESB, zbudowana w oparciu o JBI i wydana na licencji Apache, od wersji 4 ServiceMix wykorzystuje OSGi do uproszczenia podziału aplikacji na komponenty. 	
Architekturę ServiceMix'a można podzielić na 3 warstwy:
\begin{enumerate}
	\item Warstwa jądra
	\item Warstwa serwisów
	\item Warstwa aplikacji
\end{enumerate}  
Każda z tych warstw ma inne zadania i odpowiada za inne czynności.
\begin{itemize}
	\item Warstwa jądra - bazuje na Apache Karaf czyli implementacji OSGi będącą lekkim kontenerem do którego można wdrożyć różne komponenty i aplikacje. Warstwa ta wspólpracuje z warstwą serwisów w celu stworzenia, skoordynowania, utrzymania i zarządzania logowaniem, bezpieczeństwem oraz transakcjami. Najważniejsze funkcje dostarczane przez tą warstwę to:
	\begin{itemize}
		\item Osadzanie - umożliwia zarówno manualne jak i automatyczne osadzanie bibliotek
		\item Kontener OSGi - ServiceMix 4 wspiera 2 różne kontenery OSGi a mianowicie Eclipse Equinox i Apache Felix
		\item Wstrzykiwania zależnośći - wykorzystuje 2 różne frameworki:
			\begin{itemize}
				\item Blueprint
				\item Spring DI
			\end{itemize}   
		\item Automatyczna konfiguracja - dokonując zmian w pliku z właściwościami można dokonać zmian "w locie", bez restartowania serwera
		\item Bezpieczeństwo - framework odpowiadający za bezpieczeństwo bazuje na JAAS, dostarcza kilka różnych, odizolowanych poziomów:
			\begin{itemize}
				\item kontenera OSGi
				\item wbudowanej instancja serwisu wiadomości
				\item osadzonych instancji serwisow router'a i integracyjnego
			\end{itemize} 
		\item Logowanie - dynamiczne logowanie wspierające różne interfejsy takie jak: JCL, SLF4J, Avalon, łatwo konfigurowalne poprzez pliki z właściwościami
		\item Konsola - umożliwia zarządzanie i pełną kontrolę nad cała aplikacją
	\end{itemize}  
\end{itemize}





% ---------------------------------------------------------------------------
% ----------------------- end of thesis sub-document ------------------------
% ---------------------------------------------------------------------------