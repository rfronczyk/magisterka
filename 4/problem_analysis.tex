% this file is called up by thesis.tex
% content in this file will be fed into the main document

%: ----------------------- name of chapter  -------------------------
\chapter{Analiza problemu} % top level followed by section, subsection


%: ----------------------- paths to graphics ------------------------

% change according to folder and file names
\ifpdf
    \graphicspath{{X/figures/PNG/}{X/figures/PDF/}{X/figures/}}
\else
    \graphicspath{{X/figures/EPS/}{X/figures/}}
\fi

%: ----------------------- contents from here ------------------------


W tym rozdziale przanalizujemy przedstawione w poprzednim rozdziale przykładowe systemy pod kątem wyznaczenia części wspólnej, co pozwoli nam na uproszczenie i przyspieszenie implementacji tych systemów. W pierwszej części przedyskutujemy wymagania stawiane tym systemom, które z nich są ogólnymi wymaganiami aplikacji wykorzystujących synteze mowy, a które są szczególnymi wymaganiami danej aplikacji. W kolejnej części przedstawimy możliwe rozwiązania, przeanalizujemy je pod względem spełniania postawionych wymagań, a także przedstawimy wady i zalety każdego z nich.

\section {Analiza wymagań}

\section {Analiza dostępnych silników syntezy mowy}

Jako, że celem naszej pracy nie jest stworzenie własnego silnika syntezy mowy, lecz stworzenie systemów, które takie to silnik będą wykorzystywać w celu implementacji swojej logiki biznesowej, kolejnym krokiem jest analiza dostępnych na rynku rozwiązań syntezy mowy. 

Dostępne rozwiązania są charakteryzowane przez następujące cechy:
\begin{itemize}
	\item sposób dostępu
	\item ilość obsługiwanych języków
	\item jakość generowanego dźwięku
	\item cena
	\begin{itemize}
		\item koszty początkowe(cena zakupu)
		\item koszty użytkowania (opłata za każde użycie)
	\end{itemize}
	\item perspektywy rozwoju
	\item łatwość użycia
\end{itemize}

Większość z tych cech jest jasna i nie wymaga dalszego opisu. Jedyną cechą która nie dokońca przemawia za siebie jest sposób dostępu. Sposób dostępu określa w jaki sposób możemy uzywać danego rozwiązania, można powiedzieć, że definiuję interfejs jaki dane rozwiązanie posiada. Pod względem sposobu dostępu, silniki syntezy mowy dzielimy następująco:

\begin{itemize}
	\item rozwiązania zintegrowane  - rozwiązania wbudowane w urządzenię, pozawalające na synteze tylko we wcześniej przewdizianych przez producenta scenariuszach - 
		\begin{itemize}
			\item  Amazon Kindle
			\item  PocketBook eReader Pro
		\end{itemize}
	\item rozwiązania stacjonarne - rozwiązania pozwalające na generację dźwięku w obrębie jednego urządzenia, zazwyczaj dostępne jako odrębne aplikacje lub biblioteki
		\begin{itemize}
			\item Festival Speech Synthesis System
			\item FreeTTS
			\item Loquendo Mobile
			\item Loquendo Multimedia
		\end{itemize}
	\item rozwiązania serwerowe - gotowe rozwiązania serwerowe, wykorzystujące specjalistyczne protokoły w celu zwiększenia wydajności 
		\begin{itemize}
			\item IVONA Telecom
			\item IVONA Speech Server
			\item Loquendo Speech Server
		\end{itemize}
	\item rozwiązania SaaS
		\begin{itemize}
			\item IVONA Speech Cloud
		\end{itemize}
\end{itemize}

Sposób dostępu jednoznacznie wyznacza stopień trudności integracji rozwiązania w środowisku SOA. Na początku listy znajdują się rozwiązania, które ciężko zastosować do tworzenia rozbudowanych systemów(rozwiązania zintegrowane), natomiast na końcu rozwiązania który z łatwością można wykorzystać do budowanie wszelakiego rodzaju systemów (rozwiązania SaaS).

Podjęcię decyzji, którego rozwiązania będziemy używać, jak każda inna decyzja dotycząca architekury systemu, nie jest łatwa. Dodatkowo, zdecydowanie sie na jedno rozwiązanie bardzo często wymusi na nas wykorzystanie konkretnych technologii (np IVONA Telecom > MRCP), a także może nas pozbawić możliwości zmiany decyzji w przyszłości (przez brak standardów, wymiana jednego rozwiązania na inne moze wymagać dużego nakładu pracy).

\section {Architektura}

Znając wymagania które muszą spełniać przedstawione systemy, a także dostępne możliwości jeżeli chodzi o silniki syntezy mowy,  mozemy przejść do projektowania architektóry tych systemów.

\subsection {Wyspecjalizowane rozwiązania}
\subsection {Podejscie " hub-and-spoke"}
\subsection {Szyna} 

Przedstawienie jak to wyglada( kropki serwisy , komunikują sie z roznymi innymi serwisami, dodatokwe serisy: accounting itd)

Wymagania stawiane rozwiazaniom integracyjnym , Czy warto integrować ? (book1.pdf chapter 2)

Integracja systemow multimedialnych ??

Rozwiazania ?

Brak integracji

integracja EAI - hub and spikes (scentralizowany punk, single point of failure. słabo scalowalne)

SOA - ESB 




service activator (request od clienta)


% ---------------------------------------------------------------------------
%: ----------------------- end of thesis sub-document ------------------------
% ---------------------------------------------------------------------------

